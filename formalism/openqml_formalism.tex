\documentclass[aps,pra,10pt,twocolumn,groupedaddress,nofootinbib]{revtex4-1}

\usepackage{amsmath, amssymb, amsthm}
\usepackage{graphicx}% Include figure files
\usepackage{bm,bbm}% bold math
\usepackage{hyperref}% add hypertext capabilities
\usepackage[margin=2cm]{geometry}
\usepackage{xcolor}
\usepackage{algorithm}
\usepackage[noend]{algpseudocode}


\theoremstyle{plain}
\newtheorem{theorem}{Theorem}%[section]


\DeclareMathOperator{\re}{Re}
\DeclareMathOperator{\im}{Im}
\DeclareMathOperator{\tr}{Tr}
\DeclareMathOperator{\Ad}{Ad}
\DeclareMathOperator{\ad}{ad}
\DeclareMathOperator{\diag}{diag}  % diagonal vector of the given matrix / diagonal matrix with the given vector as diagonal
\DeclareMathOperator{\Ker}{Ker}    % kernel
\DeclareMathOperator{\Imag}{Imag}  % image
\DeclareMathOperator{\spec}{spec}  % eigenvalue spectrum
\DeclareMathOperator{\lcm}{lcm}    % least common multiple

\newcommand{\isom}{\cong} % isomorphic to
\newcommand{\conj}[1]{\overline{#1}} % complex conjugate
\newcommand{\mnorm}[1]{\ensuremath{\left\| #1 \right\|}} % matrix norm

\newcommand{\de}[2]{\frac{d #1}{d #2}}   % derivative
\newcommand{\pd}[2]{\frac{\partial #1}{\partial #2}}  % partial derivative
\newcommand{\hc}{\text{h.c.}}  % hermitian conjugate

\newcommand{\be}{\begin{equation}}
\newcommand{\ee}{\end{equation}}
\newcommand{\bpm}{\begin{pmatrix}}
\newcommand{\epm}{\end{pmatrix}}

\newcommand{\naturals}{\ensuremath{\mathbb N}}
\newcommand{\Z}{\ensuremath{\mathbb Z}}  % integer numbers
\newcommand{\Q}{\ensuremath{\mathbb Q}}  % rational numbers
\newcommand{\R}{\ensuremath{\mathbb R}}  % real numbers
\newcommand{\C}{\ensuremath{\mathbb C}}  % complex numbers
\newcommand{\K}{\ensuremath{\mathbb K}}  % any field
\newcommand{\I}{\mathbbm{1}} % vector space identity op

\newcommand{\SL}{\text{SL}} % special linear group
\newcommand{\GL}{\text{GL}} % general linear group
\newcommand{\OO}{\text{O}}  % orthogonal group
\newcommand{\SO}{\text{SO}} % special orthogonal group
\newcommand{\U}{\text{U}}   % unitary group
\newcommand{\SU}{\text{SU}} % special unitary group
\newcommand{\Sp}{\text{Sp}} % symplectic group


\newcommand{\CNOT}{\text{CNOT}}

\newcommand{\comm}[2]{\ensuremath{\left[#1, #2\right]}}             % commutator
\newcommand{\acomm}[2]{\ensuremath{\left\{#1, #2\right\}}}          % anticommutator
\newcommand{\ket}[1]{\ensuremath{\left| #1 \right \rangle}}
\newcommand{\bra}[1]{\ensuremath{\left \langle #1 \right |}}
\newcommand{\braket}[2]{\ensuremath{\left\langle #1\left|#2 \right.\right\rangle}}
\newcommand{\ketbra}[2]{\ket{#1}\bra{#2}}
%\newcommand{\vect}[1]{\ensuremath{\mathbf{#1}}}
\newcommand{\inprod}[2]{\ensuremath{\left\langle #1, #2 \right\rangle}}  % inner product
\newcommand{\lieprod}[2]{\ensuremath{\left[#1, #2\right]}}          % Lie product
\newcommand{\liealg}[1]{\ensuremath{\mathfrak{#1}}}                 % Lie algebra
\newcommand{\expect}[1]{\ensuremath{\left\langle #1 \right\rangle}} % expectation value
\newcommand{\tracep}[1]{\ensuremath{\trace\left( #1 \right)}}


\renewcommand{\a}{\hat{a}}
\newcommand{\x}{\hat{x}}
\newcommand{\p}{\hat{p}}
\newcommand{\bx}{\mathbf{x}}
\newcommand{\bp}{\mathbf{p}}
\newcommand{\bq}{\mathbf{q}}
\newcommand{\e}{\mathrm{e}}
\newcommand{\bad}{\bm{\hat a}^\dagger}
\renewcommand{\L}{\mathcal{L}}
\newcommand{\G}{\mathcal{G}}

\newcommand{\sidenote}[1]{\marginpar{\footnotesize{\textcolor{red}{-#1}}}}

\newcommand{\nathan}[1]{\textcolor{blue}{Nathan: #1}}
\newcommand{\maria}[1]{\textcolor{orange}{Maria: #1}}
\newcommand{\ville}[1]{\textcolor{purple}{Ville: #1}}


\begin{document}

\title{OpenQML - Technical manuscript}
\author{Maria Schuld}
\author{Nathan Killoran}
\email{nathan@xanadu.ai}
\author{Ville Bergholm}
\affiliation{Xanadu Inc., 372 Richmond St W, Toronto, M5V 1X6, Canada}


\date{\today}

\begin{abstract}
Compilation of formulas and conventions for the OpenQML framework.
\end{abstract}

\maketitle


\section{Basic building blocks}

\subsection{Variational circuit, layers and gates}

We are given a \textbf{variational circiut}
\[ U (\theta[, x]): \rho_0 \mapsto \tr \{ \rho \hat{O_i} \} := E_i. \]
The circuit takes a ground or vacuum state $\rho_0 = \ketbra{0}{0}$
and maps it to the expectation value of an operator $\hat{O}_i$ from a
set of operators $\mathcal{O} = \{\hat{O}_1 \cdots \hat{O}_{N_O}\}$
with respect to the final state $\rho$ of the circuit. These operators
correspond to measurements that are easy to make on the device, for
example when $\hat{O}$ is the Pauli operator $\sigma_z$ that
corresponds to a computational basis measurement, or the quadrature
operator $\x$ corresponding to homodyne detection in CV systems. The
circuit can also potentially take an input $x$, which we treat as a
placeholder that can be replaced with values during training.


The variational circuit can be composed of \textbf{layers}
\[ U (\theta) = \L_{N_L} \cdots \L_{1}\]
which repeat a certain architecture. A layer $\L_l$, $l=1,...,N_L$, consists of a series of \textbf{gates}
\[\L_l = \G^l_{N_G}(\theta^l_{N_G}) \cdots \G^l_1(\theta^l_1). \]
Each gate is parametrised by a set of parameters
$\theta^i_j$, $i = 1,...,N_L$, $j=1,...,N_G$,
which can be empty if the gate is not parametrised.

We propose to use the following \textbf{elementary gate sets}:

\subsubsection{Qubit architectures}

For qubit architectures we define three parametrized gates,
the elementary rotations around the $x$, $y$ and~$z$ axes:
\begin{eqnarray*}
        R_x(\alpha) &=& \e^{-i\alpha \sigma_x/2} =
        \begin{pmatrix}
          \cos \frac{\alpha}{2} & -i \sin \frac{\alpha}{2}\\
          -i \sin \frac{\alpha}{2} & \cos \frac{\alpha}{2}
        \end{pmatrix}\\
        R_y(\beta) &=& \e^{-i\beta \sigma_y/2} =
        \begin{pmatrix}
          \cos \frac{\beta}{2} & -\sin \frac{\beta}{2}\\
          \sin \frac{\beta}{2} & \cos \frac{\beta}{2}
        \end{pmatrix}\\
        R_z(\gamma) &=& \e^{-i\gamma \sigma_z/2}=
        \begin{pmatrix}
          \e^{-i \frac{\gamma}{2}} & 0\\
          0 & \e^{i \frac{\gamma}{2}}
        \end{pmatrix}\\
\end{eqnarray*}
Any $\SO(3)$ rotation can be expressed using three Euler angles, that
is, three rotations around two orthogonal axes. A standard axis choice is~$zyz$.
Due to the isomorphism $\SU(2)/\Z_2 \isom \SO(3)$ we obtain a similar decomposition
for arbitrary single-qubit gates~$U$:
\begin{align}
\label{eq:app:euler}
\notag
U &= R_z(\gamma) R_y(\beta) R_z(\alpha)\\
&=
\bpm
e^{-i(\alpha+\gamma)/2} \cos(\beta/2) & -e^{i(\alpha-\gamma)/2} \sin(\beta/2)\\
e^{-i(\alpha-\gamma)/2} \sin(\beta/2) &  e^{i(\alpha+\gamma)/2} \cos(\beta/2)
\epm.
\end{align}

Together with the $\CNOT$ gate this gate set is universal, i.e. any $\SU(2^n)$
gate can be expanded into a finite sequence of $\SU(2)$ and $\CNOT$ gates~\cite{barenco1995}.


Additionally our elementary gate set includes the following common fixed gates:
TODO




\subsubsection{CV architectures}

Displacement, squeezing, phase rotation, beam splitter, Kerr nonlinearity, cubic phase gate. See SF conventions.

\subsection{Cost function}

The \textbf{cost function} depends on the expectation vales $\{E_i\}$,
\[C(\theta[, \mathcal{D}]) = g(\{E_i\}),\]
and potentially on a data set $\mathcal{D}$ which either consists of
multiple inputs, $\mathcal{D} = \{x\}$, or input-output pairs,
$\mathcal{D} = \{(x ,y)\}$. We assume that both $x$ and $y$ are real
finite vectors or scalars.

For the following we will also assume that the cost function is a sum of functions of the expectations,
\be
C(\theta[, \mathcal{D}]) = \sum_i g_i(\{E_i\}),
\label{eq:cost}
\ee
which is true for the standard use-cases of variational circuits.

\color{black!80!white}
Examples of cost functions are:

\paragraph{Expectation cost for a variational eigensolver}

The cost function of a variational eigensolver is a weighed sum of expectation values,
\[ C(\theta) = \sum_i h_i E_i(\theta). \]
Usually, these are expectations of Pauli operators, and their sum is the energy expectation,
\[\langle H \rangle = \sum\limits_{i, \alpha} h^i_{\alpha} \langle\sigma^i_{\alpha}\rangle + \sum\limits_{\substack{i,j\\ \alpha, \beta}} h^{ij}_{\alpha, \beta} \langle \sigma^i_{\alpha}\sigma^j_{\beta}\rangle + \hdots,\]
where $i,j$ denote the qubits that the Paulis act on, and
$\alpha, \beta = x,y,z$ sum over all different combinations of Pauli operators.

\paragraph{Maximum likelihood cost for a generative model}
If the goal is to increase the likelihood of measuring a basis state
$\ket{x}$ that represents an input in the data set
$\mathcal{D} = \{x\}$, the expectations we are interested in are
\be
E_x(\theta) =  \tr\{\rho(\theta) \ketbra{x}{x}  \},
\ee
and we can use maximum likelihood to define
\[C(\theta, \mathcal{D}) = \sum_{x \in \mathcal{D}}\log E_x(\theta) .\]
Note that in this unsupervised learning task the data enters the cost function directly, and the circuit is data-independent.


\paragraph{Square loss cost for supervised learning}
If we interpret the expectation $E_x(\theta) = \tr \{ \rho(\theta, x)
\; \sigma_z \}$ of the computational basis state of a designated qubit
for an input $x$ as the prediction of a quantum classifier, the square
loss cost function reads
\[ C(\theta, \mathcal{D}) = \sum_{(x,y) \in \mathcal{D}} |E_x(\theta)-y|^2. \]

\color{black}


\section{Gradient computation}

In this section we explore issues related to computing gradients of variational quantum circuits.


\subsection{One-parameter subgroups}

A one-parameter subgroup of the topological group~$G$ is defined by a continuous group homomorphism
\be
\phi: \R \to G,
\ee
where $\R$ is a group with respect to addition.
The homomorphism property means that
\be
\phi(a)\phi(b) = \phi(a+b) \quad \forall a,b \in \R,
\ee
which immediately yields
$\phi(0) = e$ and $\phi(-a) = \phi(a)^{-1}$,
where $e$~is the identity element of~$G$.
$\Imag \phi$ is always a subgroup of~$G$.

The kernel of the homomorphism is defined as
\be
\Ker \phi = \{a \in \R | \phi(a) = e\}.
\ee

\begin{theorem}
  \label{th:1psg}
One-parameter subgroups are either isomorphic to~$\R$, or their parametrization~$\phi$ is periodic.

\begin{proof}
$\phi$ is an injection iff $\Ker \phi = \{0\}$ (trivial kernel).
In this case the subgroup $\phi(\R) < G$ is isomorphic to~$\R$.
Let us next assume that the kernel is nontrivial, i.e. for some $p \neq 0$ we have
$\phi(p) = e$.
% p \in \Ker \phi \implies n*p \in \Ker \phi
This is equivalent to
\be
\phi(a+p) = \phi(a)\phi(p) = \phi(a) \quad \forall a \in \R,
\ee
and the map~$\phi$ is thus periodic. Let us use $\tilde{p}>0$
to denote the period, i.e.
$\tilde{p} = \min \{|p| \neq 0 | p \in \Ker \phi\}$.
Now the kernel is seen to be $\Ker \phi = \{n \tilde{p} | n \in Z\}$.
\end{proof}
\end{theorem}

\subsection{Finite-dimensional Hilbert spaces}

Assume we are given a traceless hermitian $n \times n$ generator~$G$, with the eigendecomposition
\be
G = V \diag(\lambda_1, \ldots, \lambda_n) V^\dagger.
\ee
The tracelessness yields $\sum_k \lambda_k = 0$.
$G$~generates a one-parameter subgroup of the unitary group~$SU(n)$
with the homomorphism/parametrization
\be
U(\theta) = \exp(-iG\theta)
= V \diag(e^{-i\lambda_1 \theta}, \ldots, e^{-i\lambda_n \theta}) V^\dagger,
\ee
where $\theta \in \R$. Using Th.~\ref{th:1psg} we find that
\begin{align}
  & U(\theta) \: \text{is periodic with period} \: \tilde{\theta} > 0\\
  \iff \: & \Ker U = \{n \tilde{\theta} | n \in Z\}\\
  \iff \: & \lambda_k \tilde{\theta} = n_k 2 \pi \quad \forall k, \: \text{where} \: n_k \in \Z\\
  \iff \: & \frac{\lambda_i}{\lambda_j} = \frac{n_i}{n_j} \quad \forall i,j.
\end{align}

Let us extend the definition of least common multiple to real numbers:
\be
\lcm(a_1, a_2, \ldots) := \min \{q | \forall i \exists n_i \in \Z: a_i n_i = q>0\}.
\ee
Such a number exists iff $a_i$ are rational multiples of each other.
Now $\tilde{\theta} = 2\pi \: \lcm \{\lambda^{-1} \:|\: \lambda \in \spec(G) \setminus \{0\}\}$.



Qubits:
$G$~has only two eigenvalues, and the tracelessness guarantees that $\lambda_2 = -\lambda_1$.
Thus $U(\theta)$ is always periodic with $\tilde{\theta} = 2\pi/|\lambda_1|$.

Qudits:
Now almost all generators~$G$ have eigenvalues that are not rational multiples of each other.
These~$G$ generate a one-parameter subgroup of $U(d)$ isomorphic to~$\R$.
However, some generators~$G$ always generate a periodic subgroup.


\subsection{Gradient trick}

Let us assume $G^2 = r^2\I$.
This is equivalent to $\lambda_i = \pm r$.
Now we may write the Taylor series for the exponential
and split it into a cosine and a sine series:
\begin{align*}
U(\theta) &= \exp(-iG\theta) = \sum_{k=0}^\infty \frac{(-i\theta)^k G^k}{k!}\\
&=
\sum_{k=0}^\infty \frac{(-i\theta)^{2k} G^{2k}}{(2k)!}
+\sum_{k=0}^\infty \frac{(-i\theta)^{2k+1} G^{2k+1}}{(2k+1)!}\\
&=
\I \sum_{k=0}^\infty \frac{(-1)^{k} (\theta r)^{2k}}{(2k)!}
-i G/r \sum_{k=0}^\infty \frac{(-1)^{k} (\theta r)^{2k+1}}{(2k+1)!}\\
&=
\I \cos(r \theta)
-i G/r \sin(r \theta).
\end{align*}
Let us then define the unitary
\be
W := U(\pi/(4r)) = \frac{1}{\sqrt{2}}(\I -ir^{-1}G).
\ee
This yields
\begin{align}
  \notag
\Ad_W \rho
&= W \rho W^\dagger
= \frac{1}{2}\left((\I-ir^{-1}G)\rho(\I+ir^{-1}G)\right)\\
&= \frac{1}{2}\left(\rho -ir^{-1}\ad_G \rho +r^{-2}\Ad_G \rho\right)
\end{align}
and thus we obtain the convenient formula
\be
r(\Ad_W-\Ad_{W^\dagger}) = -i\ad_G.
\ee


Qubits:
This trick works for arbitrary single-qubit gates, generated by a linear combination of Pauli matrices
$G = \vec{a} \cdot \vec{\sigma}/2$
where $|\vec{a}|=1$.
We obtain $G^2 = \frac{1}{4} \I$, and thus $r=1/2$.

Qudits (including the multi-qubit case):
We may use the trick for gates generated by $G$ whose
eigenvalues fulfill $\lambda_i = \pm r$.

\section{Optimization}

OpenQML deals with the optimization of the variational circuit with
regards to the cost. We always minimize the cost. Our core method is
(stochastic) gradient descent.

\subsection{Gradient descent step}
In every \textbf{step} of the optimization algorithm, the parameters are updated according to the following simple loop\\

\begin{algorithmic}[1]
\Procedure{Gradient descent step}{}
\State [sample a batch $\mathcal{D}$ from the training data]
\For {$\mu \in \theta $}
\State $\mu^{(t+1)} = \mu^{(t)} - \eta^{(t)} \partial_{\mu} C(\theta[, \mathcal{D}]) + S$
\EndFor
\EndProcedure
\end{algorithmic}

The learning rate $\eta^{(t)}$ can be adapted in each step, either
depending on the gradient or on the step number. $S$ is a potential
additional term that can add a momentum to the update. The gradient
$\partial_{\mu} C(\theta[, \mathcal{D}])$ has to be evaluated by
automatic differentiation of the cost function, which is computed
hybridly by the quantum device or simulator and a classical
computer. We therefore need to define a hybrid automatic
differentiation scheme.

\subsection{Computing the gradient}
For the cost of Eq. (\ref{eq:cost}) and neglecting the dependency on inputs $x$ for now, we have
\[\partial_{\mu} C(\theta[, \mathcal{D}]) = \sum_i \frac{d g(\{E_i(\theta)\})}{d E_i(\theta)} \partial_{\mu} E_i(\theta). \]
The first gradient $\frac{d g(\{E_i(\theta)\})}{d E_i(\theta)}$ is the
derivative of the classically computed part of the cost. We evaluate
it using standard computational tools for automatic
differentiation. The second gradient $\partial_{\mu} E_i(\theta)$ is
the derivative of the part that is computed by the quantum device. We
need something like ``quantum automatic differentiation'' to compute
these gradients.

Formally, we have
\begin{eqnarray*}
        \partial_{\mu} E_i(\theta) &=& \partial_{\mu} \tr \{\rho(\theta) \; \hat{O}_i\} \\
        &=& \tr \{ (\partial_{\mu}  \rho(\theta)) \; \hat{O}_i\}, \\
\end{eqnarray*}
and with $\rho(\theta) = U(\theta)\rho_0U^{\dagger}(\theta)$ and the product rule of differentiation, we get
\[\partial_{\mu}  \rho(\theta) = \left(\partial_{\mu}   U(\theta)\right) \rho_0U^{\dagger}(\theta) +   U(\theta) \rho_0 \left(\partial_{\mu} U^{\dagger}(\theta)\right)   . \]
This expression contains the \textbf{circuit derivative},
$\partial_{\mu}  U(\theta)$. Assume that only the $l$th layer depends
on circuit parameter $\mu$, then
\[
\partial_{\mu}  U(\theta) =  \mathcal{L}_1^{\dagger} \cdots (\partial_{\mu} \mathcal{L}_l^{\dagger})\cdots \mathcal{L}_D^{\dagger}.
\]
We call the expression $\partial_{\mu} \mathcal{L}_d^{\dagger}$  the
\textbf{layer derivatives}. The layer is decomposed into different
gates from our universal gate set. Let $\mu$ be the parameter of the
$r$th gate $\G_r(\mu)$ (where $\G$ can also depend on other
parameters). The derivative of a layer is then the original layer, but
instead of $\G_r$ we use $\partial_{\mu} \G_r(\mu)$. This expression
is in general not a quantum gate, and in particular not necessarily a
member of our elementary gate set. We therefore have to decompose it
into a weighed sum of ``allowed gates'' $A_k(\mu)$,
\begin{equation}
        \partial_{\mu} \G(\mu) = \sum_k a_k A_k(\mu),
    \label{eq:decomposition_derivative}
\end{equation}
with complex coefficients $a_k$. This is always possible, because any
operator can be represented as a sum of unitaries, and if our
elementary gate set is universal for quantum computing, it can
represent any unitary. The derivative circuit hence becomes
\begin{eqnarray}
\partial_{\mu}  U(\theta)  &=& \sum_k a_k U_{A_k}(\theta) \\
&=& \sum_k a_k \G_1^1 \cdots A_k(\mu) \cdots \G_{N_G}^{N_L},
\end{eqnarray}
where $U_{A_k}(\theta)$ is the original circuit but with the gate $\G_r^l(\mu)$ replaced by $A_k(\mu)$. \\



An example is the special case for Equation (\ref{eq:decomposition_derivative}) when $\G(\mu) = \e^{i\mu H}$ for a generator $H$. Then, formally,
\[
\partial_{\mu}\G(\mu) = i H \e^{i\mu H}.
\]
Since $H$ might not be a unitary operator, we have to decompose it into unitaries $H_k$ via $H = \sum_k a_k H_k$, so that
\[
\partial_{\mu}\G(\mu) = \sum_k i \underbrace{H_k \e^{i\mu H}}_{A_k}.
\]

Summarizing the above, we get the formulae
\begin{eqnarray*}
        \partial_{\mu} E_i(\theta) &=& \tr \{ \sum_k \left[ a_k  U_{A_k}(\theta) \rho_0 U^{\dagger}(\theta) + h.c. \right] \; \hat{O}_i\}, \\
        &=& \sum_k \tr \{  \left[ a_k  U_{A_k}(\theta) \rho_0 U^{\dagger}(\theta) + h.c. \right] \; \hat{O}_i\}\\
        &=& \sum_k \tr \{   a_k  U_{A_k}(\theta) \rho_0 U^{\dagger}(\theta) \hat{O}_i + h.c. \}.
\end{eqnarray*}

Altogether, the quantum computer has to evaluate the terms
\[ \tr \{  U_{A_k}(\theta) \rho_0 U^{\dagger}(\theta)  \hat{O} \},\]
and
\[ \tr \{  U(\theta) \rho_0 U_{A_k}^{\dagger}(\theta)  \hat{O} \},\]
from which we can construct the derivative of the expectation classically. For this, we have to prepare a quantum state of the form
\[ \mathcal{A} \rho_0 \mathcal{B}, \]
where $ \mathcal{A}, \mathcal{B}$ are the derivative and the original variational circuit. \\

\maria{Nathan, you mentioned that you know a trick of how we can implement this easily? So far I am very sceptical that this is simple. I only know the following trick:
Decompose $\hat{O}$ into a sum of unitaries $B_s$,
\[\hat{O} = \sum_s b_s B_s,\]
so that we have to implement
\[ \tr \{  U(\theta) \rho_0 U_{A_k}^{\dagger}(\theta)  B_s \}.\]
Define $A=U(\theta)$ and $D=U_{A_k}^{\dagger}(\theta)  B_s$, then prepare
\[ \frac{1}{\sqrt{2}} (\ket{0} \otimes A\rho_0 A^{\dagger} + \ket{1} \otimes D\rho_0 D^{\dagger}),\]
and apply a Hadamard on the ancilla. The probability of measuring the ancilla in $0$ is
\[p(0) = \frac{1}{2} + \frac{1}{2} \tr \{ A\rho_0 A^{\dagger} D\rho_0 D^{\dagger} +D\rho_0 D^{\dagger} A\rho_0 A^{\dagger}  \} \]
For this, the quantum device needs ancilla qubits, and $A_k$ and $B_s$ have to be implemented conditioned on the ancilla. }

%\bibliographystyle{unsrt}
\bibliography{openqml_formalism}


\end{document}
