\documentclass[amsmath,amssymb,aps,pra,10pt,twocolumn,groupedaddress,nofootinbib]{revtex4-1}

\usepackage{graphicx}% Include figure files
\usepackage{bm,bbm}% bold math
\usepackage{hyperref}% add hypertext capabilities
\usepackage[margin=2cm]{geometry}
\usepackage{xcolor}
\usepackage{algorithm}
\usepackage[noend]{algpseudocode}

\newcommand{\ket}[1]{| #1 \rangle} % for Dirac bras
\newcommand{\bra}[1]{\langle #1 |} % for Dirac kets
\newcommand{\braket}[2]{\langle #1 | #2 \rangle} % for Dirac brackets
\newcommand{\ketbra}[2]{|#1\rangle\langle#2|}
\renewcommand{\a}{\hat{a}}
\newcommand{\ad}{\a^\dagger}
\newcommand{\x}{\hat{x}}
\newcommand{\p}{\hat{p}}
\newcommand{\bx}{\mathbf{x}}
\newcommand{\bp}{\mathbf{p}}
\newcommand{\bq}{\mathbf{q}}
\newcommand{\e}{\mathrm{e}}
\newcommand{\tr}{\mathrm{tr}}
\newcommand{\bad}{\bm{\hat a}^\dagger}
\renewcommand{\L}{\mathcal{L}}
\newcommand{\G}{\mathcal{G}}

\newcommand{\sidenote}[1]{\marginpar{\footnotesize{\textcolor{red}{-#1}}}}

\newcommand{\nathan}[1]{\textcolor{blue}{Nathan: #1}}
\newcommand{\maria}[1]{\textcolor{orange}{Maria: #1}}
\newcommand{\ville}[1]{\textcolor{purple}{Ville: #1}}


\begin{document}

\title{OpenQML - Technical manuscript}
\author{Xanadu}
\date{\today}

\begin{abstract}
Compilation of formulas and conventions for the OpenQML framework.
\end{abstract}

\maketitle


\section{Basic building blocks}

\subsection{Variational circuit, layers and gates}

We are given a \textbf{variational circiut} 
\[ U (\theta[, x]): \rho_0 \rightarrow \tr \{ \rho \hat{O_i} \} := E_i. \]
The circuit takes a ground or vacuum state $\rho_0 = \ketbra{0}{0}$ and maps it to the expectation value of an operator $\hat{O}_i$ from a set of operators $\mathcal{O} = \{\hat{O}_1 \cdots \hat{O}_{N_O}\}$ with respect to the final state $\rho$ of the circuit. These operators correspond to measurements that are easy to make on the device, for example when $\hat{O}$ is the Pauli operator $\sigma_z$ that corresponds to a computational basis measurement, or the quadrature operator $\x$ corresponding to homodyne detection in CV systems. The circuit can also potentially take an input $x$, which we treat as a placeholder that can be replaced with values during training.\\

The variational circuit can be composed of \textbf{layers} 
\[ U (\theta) = \L_{N_L} \cdots \L_{1}\]
which repeat a certain architecture. A layer $\L_l$, $l=1,...,N_L$, consists of a series of \textbf{gates}
\[\L_l = \G^l_{N_G}(\theta^l_{N_G}) \cdots \G^l_1(\theta^l_1). \]
Each gate is parametrised by a set of parameters $\theta^i_j$, $i = 1,...,N_L$, $j=1,...,N_G$, which can be empty if the gate is not parametrised. \\

We propose to use the following \textbf{elementary gate sets}:

\paragraph{Qubit architectures}

Three rotations 

\begin{eqnarray*}
	R_z(\alpha) &=& \e^{-\alpha \sigma_x /2} = 
	\begin{pmatrix} 
	\cos \frac{\alpha}{2} & -i \sin \frac{\alpha}{2}\\
	-i \sin \frac{\alpha}{2} & \cos \frac{\alpha}{2}
	\end{pmatrix}\\
	R_z(\alpha) &=& \e^{-\alpha \sigma_y /2} = 	
	\begin{pmatrix} 
	\cos \frac{\alpha}{2} & -\sin \frac{\alpha}{2}\\
    \sin \frac{\alpha}{2} & \cos \frac{\alpha}{2}
	\end{pmatrix}\\
	R_z(\alpha) &=& \e^{-\alpha \sigma_z /2}=
	\begin{pmatrix} 
	\e^{-i \frac{\alpha}{2}} & 0\\
	0 & \e^{i \frac{\alpha}{2}}
	\end{pmatrix}\\ 
\end{eqnarray*}

and the $\mathrm{CNOT}$ gate.

\paragraph{CV architectures}

Displacement, Squeezing, Phase rotation, beam splitter, Kerr nonlinearity, Cubix phase gate. See SF conventions.

\subsection{Cost function}

The \textbf{cost function} depends on the expectation vales $\{E_i\}$,
\[C(\theta[, \mathcal{D}]) = g(\{E_i\}),\]
and potentially on a data set $\mathcal{D}$ which either consists of multiple inputs, $\mathcal{D} = \{x\}$, or input-output pairs, $\mathcal{D} = \{(x ,y)\}$. We assume that both $x$ and $y$ are real finite vectors or scalars.\\

For the following we will also assume that the cost function is a sum of functions of the expectations,
\begin{equation}
	C(\theta[, \mathcal{D}]) = \sum_i g_i(\{E_i\}),
	\label{eq:cost}
\end{equation}
which is true for the standard use-cases of variational circuits.

\color{black!80!white}
Examples of cost functions are:

\paragraph{Expectation cost for a variational eigensolver}

The cost function of a variational eigensolver is a weighed sum of expectation values,
\[ C(\theta) = \sum_i h_i E_i(\theta). \]
Usually, these are expectations of Pauli operators, and their sum is the energy expectation,
\[\langle H \rangle = \sum\limits_{i, \alpha} h^i_{\alpha} \langle\sigma^i_{\alpha}\rangle + \sum\limits_{\substack{i,j\\ \alpha, \beta}} h^{ij}_{\alpha, \beta} \langle \sigma^i_{\alpha}\sigma^j_{\beta}\rangle + \hdots,\]
where $i,j$ denote the qubits that the Paulis act on, and $\alpha, \beta = x,y,z$ sum over all different combinations of Pauli operators. 

\paragraph{Maximum likelihood cost for a generative model}
If the goal is to increase the likelihood of measuring a basis state $\ket{x}$ that represents an input in the data set $\mathcal{D} = \{x\}$, the expectations we are interested in are 
\[E_x(\theta) =  \tr\{\rho(\theta) \ketbra{x}{x}  \},\]
and we can use maximum likelihood to define
\[C(\theta, \mathcal{D}) = \sum_{x \in \mathcal{D}}\log E_x(\theta) .\]
Note that in this unsupervised learning task the data enters the cost function directly, and the circuit is data-independent.


\paragraph{Square loss cost for supervised learning}
If we interpret the expectation $E_x(\theta) = \tr \{ \rho(\theta, x) \; \sigma_z \}$ of the computational basis state of a designated qubit for an input $x$ as the prediction of a quantum classifier, the square loss cost function reads 
\[ C(\theta, \mathcal{D}) = \sum_{(x,y) \in \mathcal{D}} |E_x(\theta)-y|^2. \]

\color{black}

\section{Optimization}

OpenQML deals with the optimization of the variational circuit with regards to the cost. We always minimize the cost. Our core method is (stochastic) gradient descent. 

\subsection{Gradient descent step}
In every \textbf{step} of the optimization algorithm, the parameters are updated according to the following simple loop\\

\begin{algorithmic}[1]
\Procedure{Gradient descent step}{}
\State [sample a batch $\mathcal{D}$ from the training data]
\For {$\mu \in \theta $} 
\State $\mu^{(t+1)} = \mu^{(t)} - \eta^{(t)} \partial_{\mu} C(\theta[, \mathcal{D}]) + S$
\EndFor
\EndProcedure
\end{algorithmic}

The learning rate $\eta^{(t)}$ can be adapted in each step, either depending on the gradient or on the step number. $S$ is a potential additional term that can add a momentum to the update. The gradient $\partial_{\mu} C(\theta[, \mathcal{D}])$ has to be evaluated by automatic differentiation of the cost function, which is computed hybridly by the quantum device or simulator and a classical computer. We therefore need to define a hybrid automatic differentiation scheme.

\subsection{Computing the gradient}
For the cost of Eq. (\ref{eq:cost}) and neglecting the dependency on inputs $x$ for now, we have
\[\partial_{\mu} C(\theta[, \mathcal{D}]) = \sum_i \frac{d g(\{E_i(\theta)\})}{d E_i(\theta)} \partial_{\mu} E_i(\theta). \]
The first gradient $\frac{d g(\{E_i(\theta)\})}{d E_i(\theta)}$ is the derivative of the classically computed part of the cost. We evaluate it using standard computational tools for automatic differentiation. The second gradient $\partial_{\mu} E_i(\theta)$ is the derivative of the part that is computed by the quantum device. We need something like ``quantum automatic differentiation'' to compute these gradients.\\

Formally, we have
\begin{eqnarray*}
	\partial_{\mu} E_i(\theta) &=& \partial_{\mu} \tr \{\rho(\theta) \; \hat{O}_i\} \\
	&=& \tr \{ (\partial_{\mu}  \rho(\theta)) \; \hat{O}_i\}, \\
\end{eqnarray*}
and with $\rho(\theta) = U(\theta)\rho_0U^{\dagger}(\theta)$ and the product rule of differentiation, we get
\[\partial_{\mu}  \rho(\theta) = \left(\partial_{\mu}   U(\theta)\right) \rho_0U^{\dagger}(\theta) +   U(\theta) \rho_0 \left(\partial_{\mu} U^{\dagger}(\theta)\right)   . \]
This expression contains the \textbf{circuit derivative}, $\partial_{\mu}  U(\theta)$. Assume that only the $l$th layer depends on circuit parameter $\mu$, then
\[
\partial_{\mu}  U(\theta) =  \mathcal{L}_1^{\dagger} \cdots (\partial_{\mu} \mathcal{L}_l^{\dagger})\cdots \mathcal{L}_D^{\dagger}. 
\]
We call the expression $\partial_{\mu} \mathcal{L}_d^{\dagger}$  the \textbf{layer derivatives}. The layer is decomposed into different gates from our universal gate set. Let $\mu$ be the parameter of the $r$th gate $\G_r(\mu)$ (where $\G$ can also depend on other parameters). The derivative of a layer is then the original layer, but instead of $\G_r$ we use $\partial_{\mu} \G_r(\mu)$. This expression is in general not a quantum gate, and in particular not necessarily a member of our elementary gate set. We therefore have to decompose it into a weighed sum of ``allowed gates'' $A_k(\mu)$, 
\begin{equation} 
	\partial_{\mu} \G(\mu) = \sum_k a_k A_k(\mu),
    \label{eq:decomposition_derivative}
\end{equation}
with complex coefficients $a_k$. This is always possible, because any operator can be represented as a sum of unitaries, and if our elementary gate set is universal for quantum computing, it can represent any unitary. The derivative circuit hence becomes
\begin{eqnarray}
\partial_{\mu}  U(\theta)  &=& \sum_k a_k U_{A_k}(\theta) \\
&=& \sum_k a_k \G_1^1 \cdots A_k(\mu) \cdots \G_{N_G}^{N_L},
\end{eqnarray} 
where $U_{A_k}(\theta)$ is the original circuit but with the gate $\G_r^l(\mu)$ replaced by $A_k(\mu)$. \\



An example is the special case for Equation (\ref{eq:decomposition_derivative}) when $\G(\mu) = \e^{i\mu H}$ for a generator $H$. Then, formally,
\[
\partial_{\mu}\G(\mu) = i H \e^{i\mu H}. 
\]
Since $H$ might not be a unitary operator, we have to decompose it into unitaries $H_k$ via $H = \sum_k a_k H_k$, so that 
\[
\partial_{\mu}\G(\mu) = \sum_k i \underbrace{H_k \e^{i\mu H}}_{A_k}. 
\]

Summarizing the above, we get the formulae
\begin{eqnarray*}
	\partial_{\mu} E_i(\theta) &=& \tr \{ \sum_k \left[ a_k  U_{A_k}(\theta) \rho_0 U^{\dagger}(\theta) + h.c. \right] \; \hat{O}_i\}, \\ 
	&=& \sum_k \tr \{  \left[ a_k  U_{A_k}(\theta) \rho_0 U^{\dagger}(\theta) + h.c. \right] \; \hat{O}_i\}\\
	&=& \sum_k \tr \{   a_k  U_{A_k}(\theta) \rho_0 U^{\dagger}(\theta) \hat{O}_i + h.c. \}.
\end{eqnarray*}

Altogether, the quantum computer has to evaluate the terms 
\[ \tr \{  U_{A_k}(\theta) \rho_0 U^{\dagger}(\theta)  \hat{O} \},\]
and
\[ \tr \{  U(\theta) \rho_0 U_{A_k}^{\dagger}(\theta)  \hat{O} \},\]
from which we can construct the derivative of the expectation classically. For this, we have to prepare a quantum state of the form
\[ \mathcal{A} \rho_0 \mathcal{B}, \]
where $ \mathcal{A}, \mathcal{B}$ are the derivative and the original variational circuit. \\

\maria{Nathan, you mentioned that you know a trick of how we can implement this easily? So far I am very sceptical that this is simple. I only know the following trick:
Decompose $\hat{O}$ into a sum of unitaries $B_s$,
\[\hat{O} = \sum_s b_s B_s,\]
so that we have to implement
\[ \tr \{  U(\theta) \rho_0 U_{A_k}^{\dagger}(\theta)  B_s \}.\]
Define $A=U(\theta)$ and $D=U_{A_k}^{\dagger}(\theta)  B_s$, then prepare
\[ \frac{1}{\sqrt{2}} (\ket{0} \otimes A\rho_0 A^{\dagger} + \ket{1} \otimes D\rho_0 D^{\dagger}),\]
and apply a Hadamard on the ancilla. The probability of measuring the ancilla in $0$ is
\[p(0) = \frac{1}{2} + \frac{1}{2} \tr \{ A\rho_0 A^{\dagger} D\rho_0 D^{\dagger} +D\rho_0 D^{\dagger} A\rho_0 A^{\dagger}  \} \]
For this, the quantum device needs ancilla qubits, and $A_k$ and $B_s$ have to be implemented conditioned on the ancilla. }

%\bibliographystyle{unsrt}
%\bibliography{references}


\end{document}
